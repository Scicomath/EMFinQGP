% !Mode:: "TeX:UTF-8:Hard"

\documentclass{beamer}
\usepackage[UTF8,noindent]{ctexcap}
\usepackage{amsmath}
\usepackage{physics}
\usefonttheme[onlymath]{serif}

\title{重离子碰撞中的电磁场}
\subtitle{GGP 介质的影响}
\institute{三峡大学理学院}
\author{艾鑫}
\date{\today}

\begin{document}
\maketitle

\begin{frame}{目录}
  \tableofcontents
\end{frame}

\section{简介}
\begin{frame}{简介}
  \begin{itemize}
  \item 动机: 更加准确的描述重离子碰撞中的电磁场
  \item 背景: 我们之前已经计算了重离子碰撞中的电磁场, 但是没有考虑 QGP 介质对电磁场的影响, 即我们计算的是真空中的结果
  \item 问题: 在重离子碰撞后, 真空会被激发, 形成所谓的 QGP. 若 QGP 中存在电导率$\sigma$和手征磁导率$\sigma_{\chi}$,电磁场如何计算?
  \item 方法: 采用格林函数法, 并假设$\sigma_\chi$远小于$\sigma$的情况下, 推导点电荷在介质中的电磁场公式
  \end{itemize}
\end{frame}

\section{场方程与形式解}
\begin{frame}{Maxwell 方程}
  \begin{itemize}
  \item 假设: 考虑一个无限大的均匀介质, 其导电性可以用一个恒定的电导率$\sigma$和手征磁导率$\sigma_{\chi}$来描述
  \item 在这种情况下, 总电流有三个部分:外部电流, 由电场引入的欧姆电流$\sigma\vb{E}$, 由磁场引入的手征磁电流$\sigma_{\chi} \vb{B}$
  \item Maxwell 方程为:
    \begin{align}
      \div \vb{E} &= \frac{\rho_{\text{ext}}}{\epsilon} \label{eq:max1} \\
      \div \vb{B} &= 0 \label{eq:max2} \\
      \curl \vb{E} &= - \partial_t \vb{B} \label{eq:max3} \\
      \curl \vb{B} &= \partial_t \vb{E} + \vb{J}_{\text{ext}} + \sigma \vb{E} + \sigma_{\chi} \vb{B} \label{eq:max4}
    \end{align}
    其中$\rho_{\text{ext}}$和$\vb{J}_{\text{ext}}$分别为外部电荷和外部电流密度
  \item 需要注意的是, 一般情况下, 电容率(介电常数) $\epsilon (\omega) = 1 + i \sigma / \omega$依赖于频率
  \end{itemize}
\end{frame}

\begin{frame}
  \begin{itemize}
  \item 对\eqref{eq:max3}式和\eqref{eq:max4}式取旋度, 利用公式
    \begin{equation}
      \curl (\curl \vb*{v}) = \grad(\div{\vb*{v}}) - \laplacian{\vb*{v}}
    \end{equation}
    有:
    \begin{gather}
      \curl{\curl{\vb{E}}} = - \partial_t \curl{\vb{B}} \notag \\
      \grad (\div{\vb{E}}) - \laplacian{\vb{E}} = - \partial_t (\partial_t \vb{E} + \vb{J}_{\text{ext}} + \sigma \vb{E} + \sigma_{\chi} \vb{B}) \notag \\
      \frac{1}{\epsilon}\grad\rho_{\text{ext}} - \laplacian \vb{E} = -\partial_t^2 \vb{E} -\partial_t \vb{J}_{\text{ext}} - \sigma \partial_t \vb{E} + \sigma_{\chi} \curl{\vb{E}} \notag \\
      (\laplacian - \partial_t^2 - \sigma \partial_t) \vb{E} + \sigma_{\chi} \curl{\vb{E}} = \frac{1}{\epsilon} \grad \rho_{\text{ext}} + \partial_t \vb{J}_{\text{ext}} \label{eq:B}
    \end{gather}
同理有:
\begin{equation}
  (\laplacian - \partial_t^2 - \sigma \partial_t) \vb{B} + \sigma_{\chi} \curl{\vb{B}} = - \curl{\vb{J}_{\text{ext}}} \label{eq:E}
\end{equation}
  \end{itemize}
\end{frame}

\begin{frame}
  \begin{itemize}
  \item 我们发现, \eqref{eq:B}和\eqref{eq:E}满足同样的偏微分方程:
    \begin{equation} \label{eq:3}
      \hat{L} \vb{F}(t, \vb{x}) + \sigma_{\chi} \curl{\vb{F}(t,\vb{x})} = \vb{f}(t,\vb{x})
    \end{equation}
    其中$\vb{F}(t,\vb{x})$表示$\vb{B}$或$\vb{E}$. 偏微分算符$\hat{L} = \laplacian - \partial_t^2 - \sigma \partial_t$
  \item 式\eqref{eq:3}右边的$\vb{f}(t, \vb{x})$表示源项
  \item 我们可以用矩阵把\eqref{eq:3}表示成分量形式:
    \begin{equation}
      \mqty(\hat{L} & -\sigma_{\chi}\partial_z & \sigma_{\chi}\partial_y \\
      \sigma_{\chi}\partial_z & \hat{L} & -\sigma_{\chi} \partial_x \\
      -\sigma_{\chi}\partial_y & \sigma_{\chi}\partial_x & \hat{L})
      \mqty(F_x \\ F_y \\ F_z) (t,\vb{x}) = \mqty(f_x \\ f_y \\ f_z)(t,\vb{x})
    \end{equation}
  \end{itemize}
\end{frame}

\begin{frame}
  \begin{itemize}
  \item 为了求解上述的方程, 将坐标空间变换到动量空间:
    \begin{gather}
      \vb{F}(t,\vb{x}) = \int \frac{\dd{\omega}\dd[3]{\vb{k}}}{{(2\pi)}^4} e^{-i\,\omega t + i\,\vb{k}\vdot\vb{x}} \vb{F}(\omega, \vb{k}) \\
      \vb{f}(t,\vb{x}) = \int \frac{\dd{\omega}\dd[3]{\vb{k}}}{{(2\pi)}^4} e^{-i\,\omega t + i\,\vb{k}\vdot\vb{x}} \vb{f}(\omega, \vb{k}
    \end{gather}
  \item 通过替换$\partial_t \rightarrow -i\,\omega, \nabla \rightarrow i\,\vb{k}$, 动量空间的方程为:
    \begin{equation}
      \mqty(L & -i\sigma_{\chi}k_z & i\sigma_{\chi}k_y \\
      i\sigma_{\chi}k_z & L & -i\sigma_{\chi} k_x \\
      -i\sigma_{\chi}k_y & i\sigma_{\chi}k_x & L)
      \mqty(F_x \\ F_y \\ F_z) (\omega,\vb{k}) = \mqty(f_x \\ f_y \\ f_z)(\omega,\vb{k})
    \end{equation}
    其中$L = \omega^2 + i\sigma\omega - k^2, k = |\vb{k}|$
  \end{itemize}
\end{frame}

\begin{frame}
  \begin{itemize}
  \item 可以把上式的系数矩阵写成更紧凑的形式:
    \begin{equation}
      M_{ij} = L \delta_{ij} - i\sigma_{\chi} \epsilon_{ijl}k_l
    \end{equation}
  \item 其行列式为:
    \begin{equation}
      \det M = L(L^2 - \sigma_{\chi}^2 k^2)
    \end{equation}
  \item 如果$\det M \neq 0$, 我们可以得到$M$的逆矩阵:
    \begin{equation} \footnotesize
    M^{-1} = \frac{1}{\det M} \mqty(L^2-\sigma^2_{\chi}k^2_x & iL\sigma_{chi}k_z-\sigma^2_{\chi}k_xk_y & -iL\sigma_{\chi}k_y-\sigma^2_{\chi}k_xk_z\\
                                    -iL\sigma_{\chi}k_z-\sigma^2_{\chi}k_xk_y & L^2-\sigma^2_{\chi}k^2_y & iL\sigma_{\chi}k_x-\sigma^2_{\chi}k_yk_z\\
                                    iL\sigma_{\chi}k_y-\sigma^2_{\chi}k_xk_z & -iL\sigma_{\chi}k_x-\sigma^2_{\chi}k_yk_z & L^2-\sigma^2_{\chi}k^2_z)
    \end{equation}

  \end{itemize}
\end{frame}

\begin{frame}
  \begin{itemize}
  \item 利用上式, 我们可以将$\vb{F}(\omega,\vb{k})$解出:
  \begin{equation} \footnotesize \label{eq:10}
    \vb{F}(\omega,\vb{k}) = \frac{1}{L^2-\sigma^2_{\chi}k^2}
    [L\vb{f}(\omega,\vb{k}) - i\sigma_{\chi}\vb{k}\cross \vb{f}(\omega,\vb{k})] -
    \frac{\sigma^2_{\chi}}{L(L^2-\sigma^2_{\chi}k^2)}\vb{k}[\vb{k}\vdot\vb{f}(\omega,\vb{k})]
  \end{equation}
  \item 其中源项$\vb{f}(\omega,\vb{k})$由下式给出:
    \begin{equation}
    \vb{f}(\omega,\vb{k}) =
    \begin{cases}
    -i\vb{k}\cross\vb{J}_{\text{ext}}(\omega,\vb{k}), & \text{for }\vb{B} \\
    i\vb{k}\frac{\rho_{\text{ext}}(\omega,\vb{k})}{1+i\sigma/\omega} - i\omega\vb{J}_{\text{ext}}(\omega,\vb{k}), & \text{for }\vb{E}
    \end{cases}
    \end{equation}
  \item 注意到, \eqref{eq:10}式第二项$\sim \sigma^2_{\chi}
      \vb{k}[\vb{k}\vdot\vb{f}(\omega,\vb{k})]$, 对$\vb{B}$来说是为零的,但
      是对$\vb{E}$来说, 并不为零. 对于$\vb{E}$, 这一项正比于$\sim
      [(L^2-\sigma^2_{\chi}k^2)(1+i\sigma/\omega)]^{-1}$
  \end{itemize}
\end{frame}

\begin{frame}
  \begin{itemize}
    \item 我们可以从$\vb{F}(\omega, \vb{k})$的极点(poles), 得到电磁场集体
        模式(collective modes)的色散关系$\omega(\vb{k})$
    \item 对于$\vb{B}$, 有$L^2 - \sigma^2_{\chi}k^2 = 0$, 解得:
    \begin{equation}
    \omega_{s_1s_2} = -i\sigma/2 + s_1 \sqrt{k^2 + s_2 k \sigma_{\chi} - \sigma^2/4}
    \end{equation}
    其中$s_1, s_2 = \pm 1$
    \item 对于$\vb{E}$, 还有额外的极点: $\omega = -i\sigma$
    \item 这些极点给出了在没有外源情况下的场的集体模式, 其中$\omega$和
        $\vb{k}$是独立变量
    \item 当考虑外部电荷时, 色散关系由于$\omega$和$\vb{k}$的额外的关系, 将
        会被修改. 例如:在下一节中, 我们将考虑一个沿$z$轴运动的点电荷,这将
        引入限制$\omega = vk_z$
  \end{itemize}
\end{frame}
\end{document}
